%Todo
%Table
%Colors
%Check heading structure of tableofcontents and listoffigures etc.
\documentclass[12pt,reqno,oneside]{amsbook}

\newcommand{\NOTOVERLEAF}{}

\usepackage[english]{babel}  %Language needs to be specified for accessibility; currently this is not working well with latexml


% Math
\usepackage{amsmath, amssymb, amsthm}

% If you are including computer code the following package is useful.  It is similar to the verbatim package
%\usepackage{listings} 

\usepackage{setspace}

%Graphics
\usepackage{graphicx}

%LatexML
\usepackage{latexml}
\usepackage{subcaption}
% Layout
\usepackage{geometry}
\geometry{margin=1in}

\usepackage{acronym}

% Theorem environments
\newtheorem{theorem}{Theorem}[chapter]
\newtheorem{lemma}[theorem]{Lemma}
\theoremstyle{definition}
\newtheorem{definition}[theorem]{Definition}

% Hyperref
\usepackage[pdfusetitle,hidelinks]{hyperref}

\begin{document}

% Title metadata.  Make changes here to fit your needs
\newcommand{\thetitle}{A Thesis on Early Decoding}

%Edit the following
\newcommand{\institution}{University of Illinois Chicago} 
\newcommand{\degree}{DEGREE}
\newcommand{\programname}{OFFICIAL PROGRAM NAME}
\newcommand{\committee}{Besma, Chair, Advisor \\ Name \\ Name\\ Name, Outside Members Need Affiliation Here}
\newcommand{\theauthor}{Paul Sheldon}
\newcommand{\priordegrees}{BS, University of Wisconsin, 2005\\ Another Prior degree, if Appropriate}
\newcommand{\graduationyear}{2026}
% Leave the following for metadata
\author{\theauthor}
\title{\thetitle}
\date{\today}

\frontmatter

% Custom title page
% Take care in making substantial changes to this as it needs to still work with latexml
\begin{titlepage}
    \centering
	{{\Large {\textbf{\thetitle}}}  \par}
    \vspace{2cm}
    {BY\par}
    \vspace{0.5cm}
    {\theauthor}\\
    {\priordegrees}\\
    \vspace{2cm}
	THESIS [PhD candidates may use DISSERTATION here instead]\\
	\vspace{1cm}
	Submitted as partial fulfillment of the requirements\\ \vspace{0.2cm}
	for the degree of {\degree} in {\programname} \\ \vspace{0.2cm}
	in the Graduate College of the \\  \vspace{0.2cm}
	University of Illinois Chicago, {\graduationyear}\\
	\vspace{1cm}
	Chicago, Illinois
		
	\raggedright
    \vfill
    Defense Committee: \\
    \committee \\
\end{titlepage}

\setcounter{page}{2} %This is UIC requirement, the title page is (i)

%The following is best practice if distributing both an inacccessible PDF and accessible Epub file.
\chapter*{Accessibility Statement}
\iflatexml
\noindent
This document is in an accessible EPUB format. A PDF version of this document is available at [URL] or can be obtained by contacting [contact details].
\else
\noindent
An accessible EPUB version of this document is available at [URL] or can be obtained by contacting [contact details].
\fi


\chapter*{Dedication}


Lorem ipsum dolor sit amet, consectetur adipiscing elit. Sed non risus. Suspendisse lectus tortor, dignissim sit amet, adipiscing nec, ultricies sed, dolor.

\chapter*{Acknowledgements}

Lorem ipsum dolor sit amet, consectetur adipiscing elit. Sed do eiusmod tempor incididunt ut labore et dolore magna aliqua. Ut enim ad minim veniam, quis nostrud exercitation ullamco laboris nisi ut aliquip ex ea commodo consequat.

\chapter*{Preface}


\tableofcontents
\listoffigures
\listoftables

\chapter*{List of Abbreviations}
\begin{acronym}
\acro{AWGN}{Additive White Gaussian Noise}
\acro{DoR}{Decode or reject}
\end{acronym}

\chapter{Summary}
\ac{AWGN} dui. Aenean ut eros et nisl sagittis vestibulum.



%% Text

\mainmatter

\doublespacing %other options are \onehalfspacing or \singlespacing

\chapter{Introduction}
This is the introduction chapter. We cite some classic works \cite{108245}.  


\begin{theorem}\label{thm1}
This is a theorem
\end{theorem}

We reference Theorem \ref{thm1}.

\href{https://www.uic.edu}{University of Illinois Chicago}.

% JPG, JPEG, PNG will work.  PNG and SVG do not work.
\section{Motivation}



\subsection{Historical context}
A brief overview of how the problem developed.
\begin{equation}\label{eq1}\int_0^1 f(x) dx = 2\end{equation}
How to solve \eqref{eq1}
\subsection{Open questions}
Some questions remain open for future work.

Note: I have not tested the accessibility of this table.
\begin{table}[h]
\centering
\begin{tabular}{cc}
Monkeys & Lions \\
100 & 200
\end{tabular}
\caption{Example Table}
\end{table}


\chapter{Background}
This chapter gives necessary background.

\section{Group theory}
\begin{definition}
A group is a set $G$ with a binary operation satisfying closure, associativity, identity, and inverses.
\end{definition}

\begin{theorem}
	Every finite subgroup of the multiplicative group of a field is cyclic.
\end{theorem}

\begin{proof}
This is a standard result from algebra.
\end{proof}

\chapter{Early Decoding}
\section{Introduction}
\subsection{DoR Decoding}
\ifdefined\NOTOVERLEAF
\begin{figure}[h]
\centering
\begin{subfigure}[b]{0.45\textwidth}
\centering
\includegraphics[width=\textwidth]{tikz_plots/DoRDecodingOverview_n2-1.png}
\label{fig:DoRDecodingOverview_n2}
\caption{$n=2$}
\end{subfigure}%
\hfill
\begin{subfigure}[b]{0.45\textwidth}
\centering
\includegraphics[width=\textwidth]{tikz_plots/DoRDecodingOverview_n4-1.png}
\caption{$n=4$}
\label{fig:DoRDecodingOverview_n4}
\end{subfigure}
\label{fig:DoRDecodingOverview}
\caption{Overview of DoR Decoding}
\end{figure}
\fi
\ifdefined\NOTOVERLEAF
\subsection{TIER Decoding}
\begin{figure}[h]
\centering
\begin{subfigure}[b]{0.45\textwidth}
\centering
\includegraphics[width=\textwidth]{tikz_plots/TIERDecodingOverview_wInK_n2-1.png}
\label{fig:TIERDecodingOverview_wInK_n2}
\caption{$w\in\mathcal{K}$}
\end{subfigure}%
\hfill
\begin{subfigure}[b]{0.45\textwidth}
\centering
\includegraphics[width=\textwidth]{tikz_plots/TIERDecodingOverview_wNotInK_n2-1.png}
\caption{$w\notin\mathcal{K}$}
\label{fig:TIERDecodingOverview_wNotInK_n2}
\end{subfigure}
\label{fig:TIERDecodingOverview}
\caption{Overview of TIER Decoding}
\end{figure}
\fi
\section{Results}
\ifdefined\NOTOVERLEAF
\begin{figure}[h]
\centering
\includegraphics[width=\textwidth]{tikz_plots/lambdaExplore-1.png}
\label{fig:LambdaResponses}
\caption{Respnse to change in $\lambda_0, \lambda_1$}
\end{figure}
\fi

\ifdefined\NOTOVERLEAF
\begin{figure}[h]
\centering
\includegraphics[width=\textwidth]{tikz_plots/PerformanceResults_DorToTier-1.png}
\label{fig:PerformanceResults_DorToTier}
\caption{Achievable Results}
\end{figure}
\fi

\ifdefined\NOTOVERLEAF
\begin{figure}[h]
\centering
\includegraphics[width=\textwidth]{tikz_plots/PerformanceResults_IncreaseL-1.png}
\label{fig:PerformanceResults_IncreaseL}
\caption{Achievable Results}
\end{figure}
\fi


\section{Second main result}
Another significant theorem.

\chapter{Vitae}

For UIC this can be a full CV or a short version (e.g. half a page) with previoius degrees, work experience and publications.

\appendix
\chapter{Technical Lemmas}
Here we collect some supporting lemmas.

\backmatter 
\singlespacing

\bibliographystyle{plain} % We choose the "plain" reference style
\bibliography{refs} % Entries are in the refs.bib file
%\begin{thebibliography}{9}


%\bibitem{Hartshorne}
%R.~Hartshorne, \emph{Algebraic Geometry}, Springer-Verlag, New York, 1977.

%\bibitem{Mumford}
%D.~Mumford, \emph{Abelian Varieties}, Oxford University Press, 1970.

%\bibitem{DraismaEtAl}
%J.~Draisma, E.~Horobet, G.~Ottaviani, B.~Sturmfels, and R.~R.~Thomas, 
%``The Euclidean distance degree of an algebraic variety,'' 
%\emph{arXiv:1309.0049} (2013).  
%Available at: \href{https://arxiv.org/abs/1309.0049}{https://arxiv.org/abs/1309.0049}

%\end{thebibliography}


\end{document}
